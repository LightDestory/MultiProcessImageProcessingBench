\appendix
\chapter{Appendice}
\setcounter{secnumdepth}{0}
{\section{Codice Python}\label{appendix:image_subdivision}
	
\subsection{Suddivisione in linee}

{\begin{lstlisting}[language=Python, caption={Codice per la suddivisione in linee}, label={appendix:lines_subdivision}]
def _line_image_sub_divider(self) -> list[Image.Image]:
"""
Divide the image using lines.
:return: A list of the lines that the image is divided into.
"""
	tmp_image: Image.Image = self._target_image.copy()
	if self._target_cpu_core_set == 1:
		return [tmp_image]
	sub_images: list[Image.Image] = []
	line_width = tmp_image.width // self._target_cpu_core_set
	for i in range(self._target_cpu_core_set):
		left = i * line_width
		right = left + line_width
		sub_images.append(tmp_image.crop((left, 0, right, tmp_image.height)))
return sub_images
\end{lstlisting}}

{\begin{lstlisting}[language=Python, caption={Codice per l'unione delle linee}, label={appendix:lines_merge}]
def _merge_image_lines(self, lines: list[Image.Image]) -> Image.Image:
"""
Merge the lines of an image into one image.
:param lines: The lines to merge.
:return: The merged image.
"""
	if len(lines) == 1:
		return lines[0]
	tmp_image = Image.new("RGBA", self._target_image.size)
	line_width = tmp_image.width // self._target_cpu_core_set
	for i in range(self._target_cpu_core_set):
		left = i * line_width
		right = left + line_width
		tmp_image.paste(lines[i], (left, 0, right, tmp_image.height))
	return tmp_image
\end{lstlisting}}
\newpage
\subsection{Suddivisione in quadrati}

{\begin{lstlisting}[language=Python, caption={Codice per la suddivisione in quadrati}, label={appendix:square_subdivision}]
def _square_image_sub_divider(self) -> list[Image.Image]:
"""
Divide the image using squares.
:return: A list of the squares that the image is divided into.
"""
	tmp_image: Image.Image = self._target_image.copy()
	if self._target_cpu_core_set == 1:
		return [tmp_image]
	else:
		num_parts: int = int(self._target_cpu_core_set ** 0.5)
		square_size: int = tmp_image.width // num_parts
		sub_images: list[Image.Image] = []
		for row in range(num_parts):
			for column in range(num_parts):
				x_start: int = column * square_size
				y_start: int = row * square_size
				x_end: int = x_start + square_size
				y_end: int = y_start + square_size
				sub_images.append(tmp_image.crop((x_start, y_start, x_end, y_end)))
	return sub_images
\end{lstlisting}}

{\begin{lstlisting}[language=Python, caption={Codice per l'unione dei quadrati}, label={appendix:squares_merge}]
def _merge_image_squares(self, squares: list[Image.Image]) -> Image.Image:
"""
Merge the squares of an image into one image.
:param squares: The squares to merge.
:return: The merged image.
"""
	if len(squares) == 1:
		return squares[0]
	tmp_image = Image.new("RGBA", self._target_image.size)
	square_size: int = squares[0].width
	iterator: int = int(len(squares) ** 0.5)
	for row in range(iterator):
		for column in range(iterator):
			x_start: int = column * square_size
			y_start: int = row * square_size
			x_end: int = x_start + square_size
			y_end: int = y_start + square_size
			tmp_image.paste(squares[row * iterator + column], (x_start, y_start, x_end, y_end))
	return tmp_image
\end{lstlisting}}
\newpage
\subsection{Convoluzione}
{\begin{lstlisting}[language=Python, caption={Codice per la convoluzione}, label={appendix:convolution}]
def convolve_native(target_image: Image.Image, kernel: list[list[float]], gray_scale: bool = False) -> Image.Image:
"""
Convolves the target image with the given kernel.
:param target_image: The image to convolve.
:param kernel: The kernel to convolve with.
:param gray_scale: Whether to convert the image to grayscale before convolving.
:return: The convolved image.
"""
	img_type: str = "L" if gray_scale else "RGB"
	image_width, image_height = target_image.size
	kernel_size: int = len(kernel)
	padding: int = int(kernel_size / 2)
	result_image: Image.Image = Image.new(img_type, (image_width, image_height))
	target_image = target_image.convert(img_type)
	target_pixels = target_image.load()
	result_pixels = result_image.load()
	for y in range(image_height):
		for x in range(image_width):
			x_start: int = x - padding
			x_end: int = x + padding + 1
			y_start: int = y - padding
			y_end: int = y + padding + 1
			conv_pixel: list[float, float, float] | float = 0.0 if gray_scale else [0, 0, 0]
			iterator: int = 0
			for ky in range(y_start, y_end):
				for kx in range(x_start, x_end):
					px = max(0, min(kx, image_width - 1))
					py = max(0, min(ky, image_height - 1))
					pixel = target_pixels[px, py]
					kernel_value = kernel[int(iterator / kernel_size)][iterator % kernel_size]
					if gray_scale:
						conv_pixel += pixel * kernel_value
					else:
						conv_pixel[0] += pixel[0] * kernel_value
						conv_pixel[1] += pixel[1] * kernel_value
						conv_pixel[2] += pixel[2] * kernel_value
					iterator += 1
			normalized: tuple[int, int, int] | int
			if gray_scale:
				normalized: int = max(0, min(255, int(conv_pixel)))
			else:
				normalized: tuple[int, int, int] = (
					max(0, min(255, int(conv_pixel[0]))),
					max(0, min(255, int(conv_pixel[1]))),
					max(0, min(255, int(conv_pixel[2])))
				)
			result_pixels[x, y] = normalized
	return result_image
\end{lstlisting}}
\newpage
\subsection{Generazione Kernel di Gaussian}
{\begin{lstlisting}[language=Python, caption={Codice per la generazione di un kernel di Gauss}, label={appendix:gauss}]
def generate_gauss_kernel(kernel_size: int, sigma: int) -> list[list[float]]:
"""
Generates a Gaussian kernel of the given size and sigma.
:param kernel_size: The size of the kernel.
:param sigma: The sigma value.
:return: The generated kernel.
"""
	kernel = np.fromfunction(lambda x, y: (1 / (2 * np.pi * sigma ** 2)) * np.exp(
		-((x - (kernel_size - 1) / 2) ** 2 + (y - (kernel_size - 1) / 2) ** 2) / (2 * sigma ** 2)), (kernel_size, kernel_size))
	return (kernel / np.sum(kernel)).tolist()
\end{lstlisting}}

\subsection{Operatore Morfologico d'erosione}
{\begin{lstlisting}[language=Python, caption={Codice dell'operatore morfologico erosione}, label={appendix:erosion}]
def _erosion(target_image: Image.Image, structural_element: list[list[int]]) -> Image.Image:
"""
Applies the erosion operator to an RGB image.
:param target_image: The image to apply the operator to.
:param structural_element: The structural element to use.
:return: The eroded image.
"""
	width, height = target_image.size
	kernel_size: int = len(structural_element)
	target_image = target_image.convert("RGB")
	target_pixels = target_image.load()
	result_image = Image.new("RGB", (width, height))
	result_pixels = result_image.load()
	for y in range(height):
		for x in range(width):
			pixel_chs: list[int] = [255, 255, 255]
			for ky in range(kernel_size):
				for kx in range(kernel_size):
					px = x + kx - int(kernel_size / 2)
					py = y + ky - int(kernel_size / 2)
					if 0 <= px < width and 0 <= py < height:
						pixel_r, pixel_g, pixel_b = target_pixels[px, py]
						kernel_value = structural_element[ky][kx]
						pixel_chs[0] = min(pixel_chs[0], pixel_r - kernel_value)
						pixel_chs[1] = min(pixel_chs[1], pixel_g - kernel_value)
						pixel_chs[2] = min(pixel_chs[2], pixel_b - kernel_value)
			result_pixels[x, y] = (pixel_chs[0], pixel_chs[1], pixel_chs[2])
	return result_image
\end{lstlisting}}
\newpage
\subsection{Operatore Morfologico dilatazione}
{\begin{lstlisting}[language=Python, caption={Codice dell'operatore morfologico dilatazione}, label={appendix:dilation}]
def _dilation(target_image: Image.Image, structural_element: list[list[int]]) -> Image.Image:
"""
Applies the dilation operator to an RGB image.
:param target_image: The image to apply the operator to.
:param structural_element: The structural element to use.
:return: The dilated image.
"""
	width, height = target_image.size
	kernel_size: int = len(structural_element)
	target_image = target_image.convert("RGB")
	target_pixels = target_image.load()
	result_image = Image.new("RGB", (width, height))
	result_pixels = result_image.load()
	for y in range(height):
		for x in range(width):
			pixel_chs: list[int] = [0, 0, 0]
			for ky in range(kernel_size):
				for kx in range(kernel_size):
					px = x + kx - int(kernel_size / 2)
					py = y + ky - int(kernel_size / 2)
					if 0 <= px < width and 0 <= py < height:
						pixel_r, pixel_g, pixel_b = target_pixels[px, py]
						kernel_value = structural_element[ky][kx]
						pixel_chs[0] = max(pixel_chs[0], pixel_r + kernel_value)
						pixel_chs[1] = max(pixel_chs[1], pixel_g + kernel_value)
						pixel_chs[2] = max(pixel_chs[2], pixel_b + kernel_value)
			result_pixels[x, y] = (pixel_chs[0], pixel_chs[1], pixel_chs[2])
	return result_image
\end{lstlisting}}
\newpage
\subsection{Filtro di Media}
{\begin{lstlisting}[language=Python, caption={Codice del filtro di media}, label={appendix:mean}]
def mean_filter(target_image: Image.Image, filter_size: int) -> Image.Image:
"""
Applies the mean filter to an RGB image.
:param target_image: The image to apply the operator to.
:param filter_size: The size of the filter. By default it is 3.
:return: The filtered image.
"""	
	width, height = target_image.size
	half_size: int = int(filter_size / 2)
	target_image = target_image.convert("RGB")
	target_pixels = target_image.load()
	result_image = Image.new("RGB", (width, height))
	result_pixels = result_image.load()
	for x in range(width):
		for y in range(height):
			pixel_ch: list[int] = [0, 0, 0]
			count: int = 0
			for i in range(-half_size, half_size + 1):
				for j in range(-half_size, half_size + 1):
					nx = min(max(x + i, 0), width - 1)
					ny = min(max(y + j, 0), height - 1)
					pixel = target_pixels[nx, ny]
					pixel_ch[0] += pixel[0]
					pixel_ch[1] += pixel[1]
					pixel_ch[2] += pixel[2]
					count += 1
			result_pixels[x, y] = (int(pixel_ch[0] / count), int(pixel_ch[1] / count), int(pixel_ch[2] / count))
	return result_image
\end{lstlisting}}
\newpage
\subsection{Filtro Bilaterale}
{\begin{lstlisting}[language=Python, caption={Codice del filtro bilaterale}, label={appendix:bilateral}]
def bilateral_filter(target_image: Image.Image, diameter: int, sigma_color: int, sigma_space: int) -> Image.Image:
"""
Applies the bilateral filter to an RGB image.
:param target_image: The image to apply the operator to.
:param diameter: The diameter of the neighborhood.
:param sigma_color: The sigma color value. The greater the value, the colors farther to each other will start to get mixed
:param sigma_space: The sigma space value. The greater its value, the further pixels will mix together, given that their colors lie within the sigmaColor range.
:return: The filtered image.
"""
	def bilateral_filter_gaussian(dist: float, sigma_value: int) -> float:
		return (1 / (2 * math.pi * sigma_value ** 2)) * math.exp(-(dist ** 2) / (2 * sigma_value ** 2))
	width, height = target_image.size
	target_image = target_image.convert("RGB")
	target_pixels = target_image.load()
	result_image = Image.new("RGB", (width, height))
	result_pixels = result_image.load()
	for x in range(width):
		for y in range(height):
			r_acc, g_acc, b_acc = 0.0, 0.0, 0.0
			w_acc = 0.0
			for i in range(-diameter, diameter + 1):
				for j in range(-diameter, diameter + 1):
					nx = min(max(x + i, 0), width - 1)
					ny = min(max(y + j, 0), height - 1)
					pixel = target_pixels[nx, ny]
					spatial_dist = math.sqrt(i ** 2 + j ** 2)
					range_dist = math.sqrt((pixel[0] - target_pixels[x, y][0]) ** 2 +
						(pixel[1] - target_pixels[x, y][1]) ** 2 +
						(pixel[2] - target_pixels[x, y][2]) ** 2)

					weight = bilateral_filter_gaussian(spatial_dist, sigma_space) * bilateral_filter_gaussian(range_dist, sigma_color)
					r_acc += pixel[0] * weight
					g_acc += pixel[1] * weight
					b_acc += pixel[2] * weight
					w_acc += weight
			result_pixels[x, y] = (int(r_acc / w_acc), int(g_acc / w_acc), int(b_acc / w_acc))
	return result_image
\end{lstlisting}}
\newpage
\subsection{Canny Edge Detection}
{\begin{lstlisting}[language=Python, caption={Codice del filtro di Canny}, label={appendix:canny}]
def _non_maximal_supress(gradient_magnitude: np.ndarray, gradient_direction: np.ndarray) -> np.ndarray:
"""
Suppresses the non-maximal values in the given gradient magnitude array.
:param gradient_magnitude: The gradient magnitude array.
:param gradient_direction: The gradient direction array.
:return: The suppressed gradient magnitude array.
"""
	suppressed_magnitude = np.zeros_like(gradient_magnitude)
	angles = np.rad2deg(gradient_direction)
	angles[angles < 0] += 180
	for i in range(1, gradient_magnitude.shape[0] - 1):
		for j in range(1, gradient_magnitude.shape[1] - 1):
			direction = angles[i, j]
			if (0 <= direction < 22.5) or (157.5 <= direction <= 180):  # 0 degrees
				neighbors = [gradient_magnitude[i, j - 1], gradient_magnitude[i, j + 1]]
			elif 22.5 <= direction < 67.5:  # 45 degrees
				neighbors = [gradient_magnitude[i - 1, j - 1], gradient_magnitude[i + 1, j + 1]]
			elif 67.5 <= direction < 112.5:  # 90 degrees
				neighbors = [gradient_magnitude[i - 1, j], gradient_magnitude[i + 1, j]]
			else:  # 135 degrees
				neighbors = [gradient_magnitude[i - 1, j + 1], gradient_magnitude[i + 1, j - 1]]
			if gradient_magnitude[i, j] >= max(neighbors):
				suppressed_magnitude[i, j] = gradient_magnitude[i, j]
	return suppressed_magnitude

def _double_threshold(image_array: np.ndarray, low_threshold: int, high_threshold: int) -> np.ndarray:
"""
Filters the given image array with the given thresholds.
:param image_array: The image array.
:param low_threshold:
:param high_threshold:
:return: The filtered image array.
"""
	edge_map = np.zeros_like(image_array)
	edge_map[(image_array >= high_threshold)] = 255
	edge_map[(image_array >= low_threshold) & (image_array < high_threshold)] = 127
	return edge_map

def _link_edges(image_array: np.ndarray) -> np.ndarray:
"""
Links the edges of the given image array
:param image_array: The image array.
:return: Tge image array with the edges linked.
"""
	for i in range(1, image_array.shape[0] - 1):
		for j in range(1, image_array.shape[1] - 1):
			if image_array[i, j] == 127:
				if np.max(image_array[i - 1:i + 2, j - 1:j + 2]) == 255:
					image_array[i, j] = 255
				else:
					image_array[i, j] = 0
	return image_array


def canny_edge_detector(target_image: Image.Image, gauss_size: int, sigma: int, low_threshold: int,
high_threshold: int) -> Image.Image:
"""
Applies the canny edge detection algorithm to the given image.
:param target_image: The image to apply the canny algorithm to.
:param gauss_size: The size of the gaussian kernel.
:param sigma: The sigma value for the gaussian kernel.
:param low_threshold: The low threshold for the double thresholding.
:param high_threshold: The high threshold for the double thresholding.
:return: The image with the canny algorithm applied.
"""
	target_image = target_image.convert("L")
	# Phase 1 Smoothing applying gaussian kernel
	gauss_kernel = generate_gauss_kernel(gauss_size, sigma)
	gauss_image = convolve_native(target_image, gauss_kernel)
	# Phase 2 Finding the gradients
	sobel_x_array = np.array(convolve_native(gauss_image, convolution_kernels["edge_detection (sobel-x)"], True))
	sobel_y_array = np.array(convolve_native(gauss_image, convolution_kernels["edge_detection (sobel-y)"], True))
	gradient_magnitude = np.hypot(sobel_x_array, sobel_y_array)
	gradient_direction = np.arctan2(sobel_y_array, sobel_x_array)
	# Phase 3 Non-maximal supression
	supressed = _non_maximal_supress(gradient_magnitude, gradient_direction)
	# Phase 4 Double thresholding
	thresholded = _double_threshold(supressed, low_threshold, high_threshold)
	# Phase 5 Edge tracking by linking edges
	result_array = _link_edges(thresholded)
	return Image.fromarray(result_array.astype(np.uint8))
\end{lstlisting}}