\chapter*{Abstract}\label{abstract}
Nell'\textit{image processing} si può fare uso della \textit{parallelizzazione} degli algoritmi al fine di migliorare le prestazioni computazionali e accelerare l'elaborazione delle immagini.

Con il termine "parallelizzazione" si indica la capacità dei moderni processori e delle architetture di sistema di \textit{eseguire più operazioni simultaneamente}, consentendo di sfruttare al massimo le risorse disponibili.\newline\newline
Tuttavia, va notato che la parallelizzazione non è sempre utile ed attuabile in tutti i casi di image processing. È essenziale valutare attentamente le \textit{caratteristiche} degli algoritmi prima di decidere se la parallelizzazione sia appropriata o vantaggiosa.

Alcuni algoritmi potrebbero essere semplicemente non parallelizzabili, altri parallelizzabili poco efficacemente a causa della presenza di\textit{ dipendenze dei dati dalle precedenti fasi di elaborazione}, il che limita i vantaggi della parallelizzazione.

Infine, qualora possibile, la parallelizzazione degli algoritmi di image processing non è semplice, in quanto richiede una corretta \textit{progettazione}, \textit{implementazione}, \textit{gestione della concorrenza} e \textit{sincronizzazione delle operazioni parallele} per evitare problemi come le \textit{race condition} o i \textit{conflitti di memoria}.